\documentclass[a4paper, landscape , 8pt]{scrartcl}

% use language german
\usepackage[T1]{fontenc}
\usepackage[utf8]{inputenc}
\usepackage[english, ngerman]{babel} % \selectlanguage{english} if  needed
\usepackage{lmodern} % use modern latin fonts

% format
\usepackage{geometry}
\geometry{top=1.2cm,left=0.4cm,right=0.4cm}
\textheight = 558pt

%autor
\usepackage{authblk}

%tabular
\usepackage{tabularx}

% math
\usepackage{amsmath}
\usepackage{amssymb}
\usepackage{amsfonts}
\usepackage{enumitem}

% graphic
\usepackage{graphicx}
\graphicspath{{graphic/}} 

%colors
% \usepackage{xcolor}

% Multi Columns
\usepackage{multicol}

%compact items
\setlist{topsep=0pt, leftmargin=4mm, nolistsep}
\setlength{\parindent}{0cm}

%define header and footer
\usepackage{fancyhdr}
\pagestyle{fancy}

\fancyhead[RO]{\AUTHOR| \INSTITUTE}
\fancyhead[LO]{\TITLE}
\fancyfoot[RO]{09.01.2022}
\renewcommand\headrulewidth{0pt}
\renewcommand\footrulewidth{0pt}
\headsep = -2pt
\footskip = 0pt


% Define Section Format
\usepackage{sectsty}
\usepackage{titlesec}
\usepackage[dvipsnames]{xcolor}

\titleformat{name=\section}[block]{\sffamily\normalsize}{}{0pt}{\colorsection}
\titlespacing*{\section}{0pt}{0pt}{0pt}
\newcommand{\colorsection}[1]{%
	\colorbox{sectioncolor!80}{\parbox{0.98\linewidth}{\vspace{-1pt}\color{white}\ #1 \vspace{-2pt}}}}

% Define Subsection Format
\titleformat{name=\subsection}[block]{\sffamily\small}{}{0pt}{\colorsubsection}
\titlespacing*{\subsection}{0pt}{0pt}{0pt}
\newcommand{\colorsubsection}[1]{%
	\colorbox{subsectioncolor!80}{\parbox{0.98\linewidth}{\vspace{-1pt}\color{black}\ #1 \vspace{-2pt}}}}

% Define SubSubsection Format
\titleformat{name=\subsubsection}[block]{\sffamily\small}{}{0pt}{\colorsubsubsection}
\titlespacing*{\subsubsection}{0pt}{0pt}{0pt}
\newcommand{\colorsubsubsection}[1]{%
	\colorbox{subsubsectioncolor!60}{\parbox{0.98\linewidth}{\vspace{-1pt}\color{black}\ #1 \vspace{-2pt}}}}


%define color
\definecolor{sectioncolor}{HTML}{052639}
\definecolor{subsectioncolor}{HTML}{468189}
\definecolor{subsubsectioncolor}{HTML}{8DB9B1}
\definecolor{b}{RGB}{0, 115, 192 } %Default highlight color
\definecolor{p}{RGB}{0, 43, 54} %Dark page color
\definecolor{t}{RGB}{131, 148, 150} %Dark text color
\definecolor{darkgreen}{RGB}{0,150,0}
\definecolor{dkgreen}{rgb}{0,0.6,0}
\definecolor{gray}{rgb}{0.5,0.5,0.5}
\definecolor{mauve}{rgb}{0.58,0,0.82}
\definecolor{DarkPurple}{rgb}{0.4, 0.1, 0.4}
\definecolor{DarkCyan}{rgb}{0.0, 0.5, 0.4}
\definecolor{LightLime}{rgb}{0.3, 0.5, 0.4}
\definecolor{Blue}{rgb}{0.0, 0.0, 1.0}
\definecolor{h}{RGB}{1, 101, 163}

% Code Listings
\usepackage{listings}
\usepackage{color}
\usepackage{beramono}
\usepackage{hyperref}
\hypersetup{
    colorlinks,
    linkcolor={black},
    citecolor={blue!50!black},
    urlcolor={blue!80!black}
}

\definecolor{bluekeywords}{rgb}{0,0,1}
\definecolor{greencomments}{rgb}{0,0.5,0}
\definecolor{redstrings}{rgb}{0.64,0.08,0.08}
\definecolor{xmlcomments}{rgb}{0.5,0.5,0.5}
\definecolor{types}{rgb}{0.17,0.57,0.68}

\lstdefinestyle{eclipse-style}{
	language=Java,
	showstringspaces=false,     
	basicstyle=\ttfamily\scriptsize,
	keywordstyle=\color{RoyalBlue}\ttfamily,
    stringstyle=\color{darkgreen}\ttfamily,
	commentstyle=\color{DarkPurple!60}\ttfamily,
	escapeinside={£}{£}, % latex scope within code      
	breaklines=true,
	breakatwhitespace=true,
	showspaces=false,
	showtabs=false,
	tabsize=2,
	morekeywords={length},
	numbers=none,
	numberstyle=\tiny\color{black},
	frame=none,
	aboveskip = 0em,
	belowskip = 0em
}
\lstset{
	style=eclipse-style
	% literate=  % Allow for German characters in lstlistings.
	% {Ö}{{\"O}}1
	% {Ä}{{\"A}}1
	% {Ü}{{\"U}}1
	% {ü}{{\"u}}1
	% {ä}{{\"a}}1
	% {ö}{{\"o}}1}
}

% Theorems \begin{mytheo}{title}{label}
\usepackage{tcolorbox}
\tcbuselibrary{theorems}
\newtcbtheorem[number within=section]{definiton}{Definition}%
{fonttitle=\bfseries}{def}
\newtcbtheorem[number within=section]{remember}{Merke}%
{fonttitle=\bfseries}{rem}
\newtcbtheorem[number within=section]{hint}{Hinweis}%
{fonttitle=\bfseries}{hnt}

% Front page
\newcommand{\AUTHOR}{Marius Zindel \& Mona Panchaud }
\newcommand{\INSTITUTE}{Ostschweizer Fachhochschule}

%dotted rule
\usepackage{dashrule}
\usepackage{tikz}
\usetikzlibrary{decorations.markings}
\newcommand{\drule}[3][0]{
	\tikz[baseline]{\path[decoration={markings,
	mark=between positions 0 and 1 step 2*#3
	with {\node[fill, circle, minimum width=#3, inner sep=0pt, anchor=south west] {};}},postaction={decorate}]  (0,#1) -- ++(#2,0);}}


%no indentation
\setlength{\parindent}{0cm}


% no vertical distribution
% explanation: we copy the macro columnbreak to stdcolumnbreak
% we now redefine columnbreak to always fill up null space and then execute the standard columnbreak.
\let\stdcolumnbreak\columnbreak
\renewcommand\columnbreak{\vfill\null\stdcolumnbreak}
