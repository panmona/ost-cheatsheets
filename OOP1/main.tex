%import template
\documentclass[10pt,landscape,a4paper]{article}
\usepackage{multicol}
\usepackage{hyperref}
\usepackage[utf8]{inputenc}
\usepackage{minted}
\usepackage{venndiagram}
\usepackage{cancel}
\usepackage{wasysym}
\usepackage{amssymb}
\usepackage{stmaryrd}

\usepackage[top=5mm,bottom=1mm,left=5mm,right=5mm,includehead]{geometry}

%define header and footer
\usepackage{fancyhdr}
\pagestyle{fancy}

\fancyhead[RO]{\AUTHOR\hspace{4pt}|\hspace{4pt}\INSTITUTE}
\fancyhead[LO]{\TITLE}
\usepackage[style=iso]{datetime2}
\fancyfoot[RO]{\today}
\renewcommand\headrulewidth{0pt}
\renewcommand\footrulewidth{0pt}
\headsep = -2pt
\footskip = 0pt

% Graphics
\usepackage{graphicx}
\graphicspath{{graphics/}}

% Redefine section commands to use less space
\makeatletter
\renewcommand{\section}{\@startsection{section}{1}{0mm}%
    {-1ex plus -.5ex minus -.2ex}%
    {0.5ex plus .2ex}%x
    {\normalfont\large\bfseries}}
\renewcommand{\subsection}{\@startsection{subsection}{2}{0mm}%
    {-1explus -.5ex minus -.2ex}%
    {0.5ex plus .2ex}%
    {\normalfont\small\bfseries}}
\renewcommand{\subsubsection}{\@startsection{subsubsection}{3}{0mm}%
    {-1ex plus -.5ex minus -.2ex}%
    {1ex plus .2ex}%
    {\normalfont\footnotesize\bfseries}}
\renewcommand{\@venn@radius}{4mm}
\renewcommand{\@venn@overlap}{2mm}
\renewcommand{\@venn@vgap}{1mm}
\renewcommand{\@venn@hgap}{1mm}
\makeatother

% Don't print section numbers
\setcounter{secnumdepth}{0}

\setlength{\parindent}{0pt}
\setlength{\parskip}{0pt plus 0.5ex}

\newcommand{\sqljoin}[3]{
    \begin{multicols*}{2}
        \begin{venndiagram2sets}
            #1
        \end{venndiagram2sets}
        \vfill\columnbreak
        \hspace{-8em}
        \begin{tabular}{ll}
            x & y \\
            \hline
            #2
        \end{tabular}
    \end{multicols*}
    \mint{sql}{#3}
}

\newcommand{\sql}[1]{\mintinline{sql}{#1}}



% DocInfo
\newcommand{\SUBJECT}{}
\newcommand{\TITLE}{Cheat Sheet Objektorientierte Programmierung}

\begin{document}

%import front page
% \input{./front.tex}

%do multicols
\begin{multicols*}{5}
    \setlength{\columnseprule}{0.4pt}
		\footnotesize
		
%import tableofcontents
% \input{./tableofcontents.tex}

\section{Allgemeines}
	\subsection{Primitive Datentypen}
		\begin{tabular}{p{.7cm} | p{2.1cm} | l}
			boolean&Boolescher Wert&true, false\\
			char&Textzeichen&'a','B','0','ë',...\\
			byte&Ganzzahl 8 Bit&-128 - 127\\
			short&Ganzzahl 16 Bit&-32'768 - 32'767\\
			int&Ganzzahl 32 Bit&-$2^{31}$ - $2^{31}$-1\\
			long&Ganzzahl 64 Bit&-$2^{63}$ - $2^{63}$-1, 1L\\
			float&Kommazahl 32 Bit&0.1f, 2e4f\\
			double&Kommazahl 64 Bit&0.1, 2e4\\
		\end{tabular}
		\subsubsection{Specials}
		\begin{lstlisting}
a[double - int] // cannot convert double to int
0/0 -> ArithmeticException div by zero
3.0/0 -> Double.POSITIVE_INFINITY
f + "" != f -> f ist null
g+1 == g+2 -> g=Double.POSITIVE_INFINITY
h != h -> h = Float.NaN/Double.NaN
1 % 0 -> ArithmeticException div by zero
0 % 1 -> 0
int<long<float -> Compiler-Fehler
0.0 / 0.f -> NaN (Double.NaN)
x + 1 == x -> Double.POSITIVE_INFINITY
d++ == --d -> true
Math.abs(x - y) < delta // floats vergleichen
		\end{lstlisting}

	\subsection{Call by value/reference}
		\begin{itemize}
			\item Call by Value: Kopie von Wert/Referenz
			\item Call by Reference: Werte können verändert werden
			\item Java kann nur Call by Value
		\end{itemize}
	
	\subsection{Sichtbarkeit}
		\begin{itemize}
			\item Sichtbarkeit darf erweitert werden
			\item \tiny \textcolor{b}{\texttt{public}}
				\footnotesize : Alle Klassen
			\item \tiny \textcolor{b}{\texttt{protected}}
				\footnotesize : Packages und alle Sub-Klassen
			\item \tiny \textcolor{b}{\texttt{private}}
				\footnotesize : Nur eigene und äussere \textbf{Klasse} (nicht Objekt)
			\item \tiny kein Keyword
				\footnotesize : Alle Klassen im selben Package
		\end{itemize}
	
	
	\subsection{final}
		\begin{itemize}
			\item \textbf{Attribute}: nicht veränderbar
			\item \textbf{Methoden}: nicht überschreibbar
			\item \textbf{Klassen}: nicht erbbar
		\end{itemize}
	
	\subsection{equals}
		\begin{itemize}
			\item \tiny \textcolor{b}{\texttt{equals()}} 
				\footnotesize ist standardmässig nur Referenzvergleich
			\item Inhaltlicher Vergleich: Bei \textbf{string} schon implementiert, 
			aber bei \textbf{Arrays} nicht!
			\item Immer auch hashCode() überschreiben.
		\end{itemize}
	
	\subsection{hashCode}
		\begin{itemize}
			\item Zwei gleiche Objekte = gleicher HashCode
			\item Umkehrung gilt nicht unbedingt
			\item Ungleiche Objekte können gleichen HashCode haben
		\end{itemize}
	

	\subsection{Overloading, Overriding \& Hiding}
	\begin{enumerate}
		\item Statischer Typ (links) für Overloading
		\item (null) $\rightarrow$ spezifischster wählen
		\item Dynamischer Typ (rechts) für Overriding
	\end{enumerate}
	\textbf{Overloading:} Gleiche Methode wird mit anderem Typ überschrieben.\\
	\textbf{Overriding:} Gleiche Methode wird 1:1 überschrieben. (@Override)
	
	Beachte noch folgendes Beispiel:
	\begin{lstlisting}
class Base { 
	public String name = "Base";
}
class Sub extends Base {
	public String name = "Sub";
	
	void x(Base other) {
		var x = name // "Sub"
		var y = other.name // "Base"
	}
}
Base b = new Sub();
var x = b.name // "Base" !
	\end{lstlisting}
	
	\columnbreak
	
	\subsection{Imports und Packages}
		\subsubsection{Import on Demand}
		\begin{itemize}
			\item \tiny \textcolor{b}{\texttt{import drawing.figures.*}} \footnotesize
			\item Alle public-Deklarationen im Paket (Wildcard)
			\item Keine Sub-Packages davon
		\end{itemize}
		\subsubsection{Single Type Import}
		\begin{itemize}
			\item \tiny \textcolor{b}{\texttt{import drawing.figures.Rectangle}} \footnotesize
			\item Nur Klasse Rectangle des Packages
			\item Hat Vorrang gegen{\"u}ber Import on Demand
		\end{itemize}
	
	
	\subsection{Exceptions}
	\begin{lstlisting}
String method(String s) throws Exception {
if (s == null) {
	throw new Exception("String is null"); 
} 
		\end{lstlisting}
		
		\subsubsection{Exceptions behandeln}
		\begin{lstlisting}
try {
	 /* regular code */ 
}  catch (Exception e) {
	/* error handling */
} finally {
	/* e.g. cleanup */ 
}
		\end{lstlisting}
		\subsubsection{Eigene Exceptions}
		\begin{lstlisting}
class StringClipException extends Exception {
	public StringClipException() {}
	public StringClipException(String message) {
		super(message); }}
//------------------in Methode:------------------
String clip(String s) throws StringClipException {
	if (s == null) {
	throw new StringClipException("no string"); } }
		\end{lstlisting}
		\subsubsection{Checked und Unchecked Exceptions}
		\textcolor{b}{Checked Exceptions:}
		\begin{itemize}
			\item Exception muss behandelt werden (catch)
			\item \textbf{throws}-Deklaration im Methodenkopf
			\item Wird vom Compiler geprüft
			\item \tiny \texttt{ClassNotFoundExc., IllegalAccessExc., EigeneExc.}
		\end{itemize}
		\textcolor{b}{Unchecked Exceptions}
		\begin{itemize}
			\item Keine throws-Deklaration/Behandlung nötig
			\item Wird nicht vom Compiler geprüft
			\item \tiny \texttt{RuntimeException, NullPointerException, IndexOutOfBoundsException, Error}
		\end{itemize}
		
	
\section{Vererbung}
	\subsubsection{Abstract Class}
	\begin{lstlisting}
public abstract class Vehicle {
	//Vererbbare Methoden....
	public abstract void drive();
	\end{lstlisting}
	\subsubsection{Erbende Klassen von Klassen}
	\begin{lstlisting}
public class Car extends Vehicle {
	@Override
	public void drive() {
	System.out.println("Car");
	//Ausfuehren der Vatermethode wenn nicht abstract
	super.drive(); 
	\end{lstlisting}
	\subsubsection{Erben von Interfaces}
	\begin{lstlisting}
public class RegularCar implements Vehicle {...}
	\end{lstlisting}




\section{Interfaces}
	\subsection{Allgemeines}
	\begin{itemize}
		\item Facette einer Klasse
		\item Trennung Spezifikation und Implementation
		\item Lollipop oder Hierarchische Mehrfachvererbung
		\item Nur Klassen sind instanzierbar
		\item Schnittstellen sind aber als statische Typen verwendbar (linke Seite)
		\item Variablen in Interfaces sind Konstanten!
		\item Keine Instanzvariablen im Interface
	\end{itemize}
	\subsection{Vererbung}
		\subsubsection{Interface $\rightarrow$ Interface}
			\begin{lstlisting}
public interface Vehicle{...}
public interface LandMobile extends Vehicle{...}
			\end{lstlisting}
		\subsubsection{Interface $\rightarrow$ Klasse}
			\begin{lstlisting}
public interface Vehicle{...}
public class Car implements Vehicle{...}
			\end{lstlisting}
	
	
	\columnbreak
	
	
	\subsection{Default Methods}
	\begin{itemize}
		\item Falls Kollisionen bei Mehrfachvererbung 
		\item Ähnlich zu Mehrfachvererbung von abstrakten Klassen
		\item Keine Instanzvariablen in Schnittstelle!
		\item Methoden können in Klasse überschrieben werden
		\item Falls nicht überschrieben: Defaultmethode
	\end{itemize}
		\begin{lstlisting}
interface Vehicle {
	public default void printModel() {...method...}
		\end{lstlisting}





	\section{Collections}
	\subsection{Array}
		\begin{lstlisting}
	// possible array inits:
	var array = new int[] { 1, 2, 3};
	int array[] = { 1, 2, 3 };
	var array = new int[3]; // empty
	
		\end{lstlisting}
	
	\subsection{List}
		\subsubsection{ArrayList}
			\begin{itemize}
				\item Duplikate möglich
				\item null ist auch einfügbar
				\item Add/Remove während Iterierens $\rightarrow$ Fehler
				\item get(), set(), add(): sehr schnell
				\item remove(), contains(): langsam
			\end{itemize}
			\begin{lstlisting}
ArrayList<String> stringList = new ArrayList<>();
List.of(array); // Create list out of array
	stringList.add("OO"); //Add Element
	stringList.add(0, "Bsys1"); //Add @ pos 0
	String x = stringList.get(1); //get @ pos 1
	stringList.set(0, "Bsys2"); //replace at pos 0
	boolean b = stringList.contains("CN1")
	stringList.remove("ICTh");
	stringList.remove(1); //remove @ pos 1
			\end{lstlisting}

		\subsubsection{LinkedList}
		\begin{itemize}
			\item Verkettete Liste der Elemente
			\item Duplikate möglich
			\item Dynamisch hinzufügbar und entfernbar
			\item Intern Doppelt verkettet
			\item add(), remove(anfang/ende) sehr schnell
			\item get(), set(), contains(), remove() langsam
		\end{itemize}
		\begin{lstlisting}
List<String> firstList = new LinkedList<>();
	//Gleiche Funktionen wie ArrayList
		\end{lstlisting}
	
	
		\subsubsection{Queue / Deques}
		\begin{itemize}
			\item Verkettete Liste der Elemente
			\item Dynamisch hinzufügbar und entfernbar
			\item LIFO oder FIFO möglich
			\item Intern Doppelt verkettet
		\end{itemize}
		\begin{lstlisting}
Deque<String> queue = new LinkedList<>();
	queue.addLast(elem);
	queue.addFirst(elem);
	queue.removeFirst(elem);
	queue.removeLast(elem);
		\end{lstlisting}

	\subsubsection{Stack}
	\begin{lstlisting}
		var stack = new Stack<E>();
		var poppedEl = stack.pop();
		var pushedItem = stack.push(el);
		var isEmpty = stack.empty();
	\end{lstlisting}


	\subsection{Set}
		\begin{itemize}
			\item Für loop: enhanced for verwenden
		\end{itemize}
		\subsubsection{HashSet}
		\begin{itemize}
			\item Keine Duplikate
			\item Unsortiert 
			\item \textbf{Oft} sehr effizient
			\item In Hashtabelle gespeichert
			\item Elemente geben hashCode() konsistent zu equals()
		\end{itemize}
		\begin{lstlisting}
Set<String> otherSet = new HashSet<>();
Set<String> s = new HashSet<>(list);
	set.add(elem);
	set.addAll(list); //Collection of elem
	set.size();
	set.remove(elem);
	set.isEmpty();
	set.contains(elem)
	String[] a = (String[]) set.toArray();
		\end{lstlisting}
		\subsubsection{TreeSet}
		\begin{itemize}
			\item Keine Duplikate
			\item Sortiert
			\item \textbf{immer} effizient
			\item In Binärbäumen gespeichert
			\item Elemente implementieren Comparable und equals()
		\end{itemize}
		\begin{lstlisting}
Set<String> firstSet = new TreeSet<>();
	//Gleiche Funktionen wie HashSet
		\end{lstlisting}
	
	\subsection{Map}
		\subsubsection{TreeMap}
		\begin{itemize}
			\item Mengen von Schlüssel-Wert-Paaren
			\item Nach Schlüssel sortiert
			\item immer effizient
		\end{itemize}
		\begin{lstlisting}
Map<Integer, String> bachelors = new TreeMap<>();
	map.put(20000, "Hello");
	map.containsKey(key);
	map.containsValue(value);
	String x = map.get(20000);
	for (int number : map.keySet()) { 
		// print all keys
		System.out.println(number);
	}
	for (String value : map.values()) { 
		// print all values
		System.out.println(value);
	}
		\end{lstlisting}
		\subsubsection{HashMap}
		\begin{itemize}
			\item Mengen von Schlüssel-Wert-Paaren
			\item Unsortiert
			\item oft sehr effizient
		\end{itemize}
		\begin{lstlisting}
Map<Integer, String> map = new HashMap<>();
// Gleiche Funktionen wie TreeMap
		\end{lstlisting}


\section{Stream API}
	\subsection{Zwischenoperationen}
		\begin{itemize}
			\item \tiny \textcolor{b}{\texttt{filter(Predicate)}}
				\footnotesize Rauspicken gem. Predicate
			\item \tiny \textcolor{b}{\texttt{map(Function)}}
				\footnotesize Projizieren gemäss Function
			\item \tiny \textcolor{b}{\texttt{mapToInt(Function)}}
				\footnotesize Proji. auf primitiver Typ
			\item \tiny \textcolor{b}{\texttt{sorted()}}
				\footnotesize Sortieren
			\item \tiny \textcolor{b}{\texttt{distinct}}
				\footnotesize Duplikate werden gelöscht
			\item \tiny \textcolor{b}{\texttt{limit(long n)}}
				\footnotesize Erste n Elemente liefern
			\item \tiny \textcolor{b}{\texttt{skip(long n)}}
				\footnotesize Erste n Elemente ignorieren
		\end{itemize}
	\subsection{Terminaloperationen}
		\begin{itemize}
			\item \tiny \textcolor{b}{\texttt{forEach(Consumer)}}
				\footnotesize Pro Element anwenden
			\item \tiny \textcolor{b}{\texttt{forEachOrderd(consumer)}}
				\footnotesize Erhält die Reihenfolge der Elemente
			\item \tiny \textcolor{b}{\texttt{count()}}
				\footnotesize Anzahl Elemente als long
			\item \tiny \textcolor{b}{\texttt{min(), max()}}
				\footnotesize Mit Comparator Argument ausser schon korrekt gemapped. Als Optional.
			\item \tiny \textcolor{b}{\texttt{average()}}
				\footnotesize Nur bei int, long, double. Als Optional.
			\item \tiny \textcolor{b}{\texttt{sum()}}
				\footnotesize Nur bei int, long, double.
			\item \tiny \textcolor{b}{\texttt{findAny()}}
				\footnotesize Gibt irgendein Element als Optional
			\item \tiny \textcolor{b}{\texttt{findFirst()}}
				\footnotesize Gibt erstes Element als Optional
		\end{itemize}
	\subsection{Optional}
		\begin{itemize}
			\item \tiny \textcolor{b}{\texttt{isPresent()}}
				\footnotesize true wenn Element vorhanden ist
			\item \tiny \textcolor{b}{\texttt{isEmpty()}}
				\footnotesize true wenn \textbf{kein} Element vorhanden ist
			\item \tiny \textcolor{b}{\texttt{get()}}
				\footnotesize Gibt Element. \textbf{Exception} wenn empty.
		\end{itemize}
	\subsection{Rückumwandlung}
		\textcolor{b}{Zu Array:}
		\begin{lstlisting}
Person[] arr = peopleStream.toArray(Person[]::new);
		\end{lstlisting}
		\textcolor{b}{Zu List}
		\begin{lstlisting}
List<Person> list = peopleStream.toList();
		\end{lstlisting}	
	
		\subsection{Lambda Beispiele}
	\begin{lstlisting}
Arrays.stream(people) // Array zu stream

stream().forEach(System.out::println); // alles ausgeben

stream().sort((p1, p2) -> 
Integer.compare(p1.getAge(), p2.getAge()))

stream().filter(p -> p.getAge() >= 18)

stream().filter(p -> p instanceof House)

stream().filter(p -> p.getGender
.equals(Gender.FEMALE))

stream().map(p -> p.getName()) // to String

stream().mapToInt(i -> i).sum(); // Sum
	\end{lstlisting}
	\columnbreak % Kind of a hack
	\begin{lstlisting}
// Top 10 earner
people.stream()
.sorted(Comparator.comparing(
Person::getSalary).reversed())
.mapToInt(p -> p.getSalary())
.limit(10)


// get Max Age from People in Zurich
people.stream()
.filter(p -> p.getCity().equals("Zuerich"))
.mapToInt(p -> p.getAge())
.max()
.getAsInt();

// get Average Age of Male
var x = people.stream()
.filter((p) -> p.getGender().equals(Gender.MALE))
.mapToInt(p -> p.getAge())
.average();

// Get all female Names with 3 or less Characters
people.stream()
.filter((p) -> p.getGender().equals(Gender.FEMALE) 
&& p.getFirstName().length() <= 3)
.map(p -> p.getFirstName())

// Durchschnittsalter pro Ort (mittels Collector)
people.stream()
.collect(Collectors.groupingBy(Person::getCity, Collectors.averagingInt(Person::getAge)))
.forEach((city, age) -> 
	System.out.println(city + " " + age));

// Gruppieren
Map<Integer, List<Person>> peopleByAge = 
people.stream()
.collect(Collectors.groupingBy(Person::getAge));
		
	\end{lstlisting}
	
	\subsubsection{Stream Beispiele}
	\begin{lstlisting}
people.stream().map(p->p.name)
IntStream.range(1, 100) // Zahlen von bis 100
Stream.of(2,3,5,7) // Eigene Aufzaehlung
Stream.concat(stream1, stream2) // 1 gefolgt von 2
	\end{lstlisting}


\section{Beispiele}

	\subsubsection{Switch - Case}
	\begin{lstlisting}
int x = 5; 
switch(x) {
	case 2: 
		System.out.println("2"); 
		break;
	case 5: 
		System.out.println("5"); 
		break;
	default: 
		System.out.println("No match!"); 
}
	\end{lstlisting}

	\subsubsection{for loop}
	\begin{lstlisting}
	for(int i = 0; i < list.size(); i++) {}
	
	for(var el : list) {}
	
	for(var el : set) {}
	
	for(var el : map.entrySet()) {
		var k = el.getKey();
		var v = el.getValue();
	}
	\end{lstlisting}

	\subsubsection{Main Methode}
	\begin{lstlisting}
	public static void main(String[] args) {...}	
	\end{lstlisting}

	\subsubsection{Interface}
	\begin{lstlisting}
	public interface GraphicItem {
		Vector getLocation();
		void move(Vector delta);
		
		// A default implementation:
		default void moveToCenter() {
			var pos = getLocation();
			move(new Vector(-pos.getX(), -pos.getY()));
		}
	}
	\end{lstlisting}

	\columnbreak

	\subsubsection{Alle Wörter aus Buchstaben generieren}
		\begin{lstlisting}
Set<String> generateWords(Set<Character> input) { 
	Set<String> result = new HashSet<>();
	if (input.size() == 0) { 
		result.add(""); 
	}
	for (Character letter : input) {
		Set<Character> remainder = new HashSet<>(input); 
		remainder.remove(letter);
		for (String word : generateWords(remainder)) {
			result.add(word + letter);
		}
	}
	return result;
}
		\end{lstlisting}
	
	\subsubsection{Substring / Split}
		\begin{lstlisting}
int number = Integer.parseInt(line.substring(0, 3));
String name = line.substring(4);
String a = "1234 10"; 
String[] c = a.split(" ");
		\end{lstlisting}


	\subsubsection{Podcast}
	\begin{lstlisting}
class Chapter {
	// Constructor..., Attributes...
	public int getLength() {
		return content.length;
	}
}
public class Podcast {
	// Constructor..., Attributes...
	public int getLength() {
		int sum = 0;
		for (var chapter : chapters) {
			sum += chapter.getLength();
		}
		return sum;
	}
}
public class PodcastMaker {
	static ArrayList<byte[]> splitIntoChapters(byte[] recording, int[] chapterMarks) {
		var result = new ArrayList<byte[]>();
		var previousChapterEnd = 0;
		
		for (int nextChapterEnd : chapterMarks) {
			var nextChapter = Arrays.copyOfRange(recording, previousChapterEnd, nextChapterEnd);
			result.add(nextChapter);
			previousChapterEnd = nextChapterEnd;
		}
		
		var lastChapter = Arrays.copyOfRange(recording, previousChapterEnd, recording.length);
		result.add(lastChapter);
		return result;
	}
	static Podcast createPodcast(ArrayList<byte[]> chapters) {
		var chapterObjects = new ArrayList<Chapter>();
		for (var chapter : chapters) {
			chapterObjects.add(new Chapter(chapter));
		}
		return new Podcast(chapterObjects);
	}
}
	\end{lstlisting}

	\subsubsection{toString mit Indent}
	\begin{lstlisting}
	// Vaterklasse:
	public abstract String toString(int indent);			
	
	@Override
	public String toString() {
		return toString(0);}
	
	protected String getIndent(int length) { 
		String text = ""; 
		for (int i = 0; i < length; i++) { 
			text += " ";
		}
		return text;
	}
	// Erbende Klasse:
	@Override
	public String toString(int indent) {
		String text = getIndent(indent) + getName() +"\n";
		for (Food food : ingredients) {
			text += food.toString(indent + 1);
		}
		return text; }
	\end{lstlisting}

	\subsubsection{Erastothenes}
		\begin{lstlisting}
// The numbers in the sieve start with 2,
// so we reduce the array length accordingly.
int[] sieve = new int[PRIMES_UP_TO - 1];
for(int i = 0; i < sieve.length; i++) {
	sieve[i] = i + 2;
}

// Now let's look at each number in the array:
for(int i = 0; i < sieve.length; i++) {
	int primeNumber = sieve[i];
	
	// Has the number already been crossed out?
	if(primeNumber == -1) {
		continue;
	}
	
	int numberToCross = primeNumber * 2;
	while(numberToCross <= PRIMES_UP_TO) {
		// Since the numbers and indices differ
		// by two, we have to calculate the index:
		int indexToCross = numberToCross - 2;
		// An int[] always needs to contain a valid
		// int, we can't really cross an item, but
		// instead set a special value:
		sieve[indexToCross] = -1;
		// And we continue with the next number:
		numberToCross += primeNumber;
	}
}
		\end{lstlisting}
	
	
	\subsubsection{Custom 'Linked List'}
	\begin{lstlisting}
public class Car {
	private Car next;
	// ...
}
public class Locomotive {
	private Car first;
	// ...
}
public class Train {
	private Locomotive locomotive;
	public Train() {
		locomotive = new Locomotive();
	}
	
	public void add(String name) {
		Car first = locomotive.getFirst();
		Car newCar = new Car(name);
		newCar.setNext(first);
		locomotive.setFirst(newCar);
	}
	
	public boolean insert(int position, String name) {
		if (position == 0) {
			add(name);
			return true;
		}
		Car leadingCar = getCar(position - 1);
		if (leadingCar == null) {
			return false;
		}
		Car followingCar = leadingCar.getNext();
		Car newCar = new Car(name);
		leadingCar.setNext(newCar);
		newCar.setNext(followingCar);
		return true;
	}
	
	public void reverse() {
		Car previousCar = null;
		Car currentCar = locomotive.getFirst();
		while (currentCar != null) {
			Car nextCar = currentCar.getNext();
			currentCar.setNext(previousCar);
			previousCar = currentCar;
			currentCar = nextCar;
		}
		locomotive.setFirst(previousCar);
	}
	
	public Car removeFirst() {
		Car firstCar = locomotive.getFirst();
		if (firstCar != null) {
			locomotive.setFirst(firstCar.getNext());
		}
		return firstCar;
	}
}
	\end{lstlisting}
	
	\subsection{compareTo}
	
	\subsubsection{Compare To}
		\begin{lstlisting}
@FunctionalInterface
interface Comparable<T> {
	int compareTo(T other); }

class Person implements Comparable<Person> {
	private int age;
	@Override 
	public int compareTo(Person other) {
		return Integer.compare(age, other.age); }
	//oder
	@Override 
	public int compareTo(Person other) {
		int result = lastName.compareTo(other.lastName);
		if (result == 0) {
			result = firstName.compareTo(other.firstName); } 
		return result;}}
		//oder
		people.sort(this::compareByAge);//in Methode
		//oder
		(p1, p2) -> p1.compareTo(p2)
		\end{lstlisting}
	
	
	
	\subsubsection{Comparator}
		\begin{lstlisting}
class AgeComparator implements Comparator<Person> {
	@Override
	public int compare(Person first, Person second){
		return Integer.compare(first.getAge(), second.getAge() ); }
		\end{lstlisting}
	
	\subsubsection{Comparable}
		\begin{lstlisting}
class Person implements Comparable<Person> {
	//.....
	@Override
	public int compareTo(Person other){
		int result = lastName.compareTo(other.lastName);
		if (result == 0){ 
		 result = firstName.compareTo(other.firstName);}
		return result;}
		\end{lstlisting}

	\subsection{equals}
		
		\begin{lstlisting}
@Override
public boolean equals(Object obj) {
	if (obj == null) {
		return false;
	}
	if (getClass() != obj.getClass()) {
		return false;
	}
	if (!super.equals(obj)) {
		return false;
	}
	Person other = (Person) obj;
	return Objects.equals(getFirstName(), other.getFirstName()) 
		&& Objects.equals(getLastName(), other.getLastName());
}
		\end{lstlisting}
	
	\subsection{hashCode}
		\begin{lstlisting}
@Override
public int hashCode() {
	return Objects.hash(firstName, lastName);
}
		\end{lstlisting}
	
	
	\subsection{Enum}
	\begin{lstlisting}
public enum Weekday {
	MONDAY, TUESDAY, WEDNESDAY, THURSDAY, FRIDAY, SATURDAY, SUNDAY 
	
	public boolean isWeekend() {
		return this== SATURDAY || this== SUNDAY;
	}
}
	\end{lstlisting}


	
	\subsection{Diverses Rekursives}
	\subsubsection{Palindrom}
	\begin{lstlisting}
printPalindroms("", "", 0, 5); //in Main

void printPalindroms(String prefix, String suffix, int pos, int length) {
	if (pos == (length + 1) / 2) {
		System.out.println(prefix + suffix);
	} else {
		boolean middle = position == length / 2;
		for (char c = 'A'; c <= 'Z'; c++) {
			printPalindroms(prefix + c, middle ? 
			suffix : c + suffix, position + 1, length);
		}}}
	\end{lstlisting}

\end{multicols*}

% \input{./appendix.tex}

\end{document}