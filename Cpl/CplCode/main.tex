% Document
\documentclass[fontsize=8pt,paper=a4,paper=landscape,DIV=calc,]{scrartcl}
\usepackage[T1]{fontenc}
\usepackage{noto}
\usepackage[nswissgerman,english]{babel}
\renewcommand{\familydefault}{\sfdefault}

% Format
\usepackage[top=5mm,bottom=1mm,left=5mm,right=5mm,includehead]{geometry}
\setlength{\headheight}{\baselineskip}
\setlength{\headsep}{0mm}

\usepackage{multicol}
\setlength{\columnsep}{2mm}
\setlength{\columnseprule}{0.1pt}

% Color
\usepackage[svgnames]{xcolor}

% Math
\usepackage{amsmath}
\usepackage{amssymb}
\usepackage{amsfonts}
\newcommand*{\eq}{=}

%% Venn Diagrams
\usepackage{venndiagram}

%% Trees
%%% https://tex.stackexchange.com/a/425271
\usepackage{forest}
\forestset{
    ptree/.style={
        for tree={
            % grow'=0,
            % parent anchor=children,
            % child anchor=parent,
            grow'=east,
            parent anchor=east,
            child anchor=west,
            text width=7mm
        },
        before typesetting nodes={
            for tree={
                split option={content}{:}{content, my edge label},
            },
        },
    },
    my edge label/.style={
        if={
            > O_= {n'}{1}
        }{
            edge label={node [midway, below left, font=\tiny] {#1} }
        }{
            edge label={node [midway, above left, font=\tiny] {#1} }
        },
    }
}

% Standards
\newcommand{\rfc}[1]{\href{https://www.rfc-editor.org/rfc/rfc#1.html}{RFC#1}}
\newcommand{\ieee}[1]{\href{https://ieeexplore.ieee.org/search/searchresult.jsp?queryText=#1}{IEEExplore #1}}

% Code
\usepackage{listings}

\lstset{
   extendedchars=true,
   basicstyle=\footnotesize\ttfamily,
   tabsize=2,
   breaklines=true,
   showspaces=false,
   showtabs=false
   showstringspaces=false,
}

%% https://tex.stackexchange.com/a/536018
%% Allow for German characters in lstlistings.
\lstset{literate=
    {Ö}{{\"O}}1
    {Ä}{{\"A}}1
    {Ü}{{\"U}}1
    {ü}{{\"u}}1
    {ä}{{\"a}}1
    {ö}{{\"o}}1
}

%% Java Language definition
\lstdefinelanguage{java}{
  keywords=[1]{abstract, assert, boolean, byte, char, class, default, double,
  enum, extends, final, float, implements, import, instanceof, int, interface,
  long, native, null, package, private, protected, public, short, static,
  strictfp, super, synchronized, this, throw, throws, transient, void,
  volatile},
  keywordstyle=[1]\color{DarkBlue}\bfseries,
  keywords=[2]{if, else, while, do, try, case, catch, finally, new, break,
  continue, return, switch},
  keywordstyle=[2]\color{DarkRed}\bfseries,
  identifierstyle=\ttfamily,
  sensitive=false,
  comment=[l]{//},
  morecomment=[s]{/*}{*/},
  commentstyle=\color{DarkGray},
  stringstyle=\color{DarkGreen},
  morestring=[b]',
  morestring=[b]"
}

% Images
\usepackage{graphicx}
\graphicspath{{graphic/}}

% Links
\usepackage{hyperref}
\hypersetup{
    colorlinks=true,
    linkcolor=blue,
    filecolor=magenta,
    urlcolor=cyan,
}

% Smaller Lists
\usepackage{enumitem}
\setlist[itemize,enumerate]{leftmargin=3mm, labelindent=0mm, labelwidth=1mm, labelsep=1mm, nosep}
\setlist[description]{leftmargin=0mm, nosep}
\setlength{\parindent}{0cm}

% Smaller Titles
\usepackage[explicit]{titlesec}

%% Color Boxes
\newcommand{\sectioncolor}[1]{\colorbox{black!60}{\parbox{0.97\linewidth}{\color{white}#1}}}
\newcommand{\subsectioncolor}[1]{\colorbox{black!50}{\parbox{0.97\linewidth}{\color{white}#1}}}
\newcommand{\subsubsectioncolor}[1]{\colorbox{black!40}{\parbox{0.97\linewidth}{\color{white}#1}}}
\newcommand{\paragraphcolor}[1]{\colorbox{black!30}{\parbox{0.97\linewidth}{\color{white}#1}}}
\newcommand{\subparagraphcolor}[1]{\colorbox{black!20}{\parbox{0.97\linewidth}{\color{white}#1}}}

%% Title Format
\titleformat{\section}{\vspace{0.5mm}\bfseries}{}{0mm}{\sectioncolor{\thesection~#1}}[{\vspace{0.5mm}}]
\titleformat{\subsection}{\vspace{0.5mm}\bfseries}{}{0mm}{\subsectioncolor{\thesubsection~#1}}[{\vspace{0.5mm}}]
\titleformat{\subsubsection}{\vspace{0.5mm}\bfseries}{}{0mm}{\subsubsectioncolor{\thesubsubsection~#1}}[{\vspace{0.5mm}}]
\titleformat{\paragraph}{\vspace{0.5mm}\bfseries}{}{0mm}{\paragraphcolor{\theparagraph~#1}}[{\vspace{0.5mm}}]
\titleformat{\subparagraph}{\vspace{0.5mm}\bfseries}{}{0mm}{\subparagraphcolor{\thesubparagraph~#1}}[{\vspace{0.5mm}}]

%% Title Spacing
\titlespacing{\section}{0mm}{0mm}{0mm}
\titlespacing{\subsection}{0mm}{0mm}{0mm}
\titlespacing{\subsubsection}{0mm}{0mm}{0mm}
\titlespacing{\paragraph}{0mm}{0mm}{0mm}
\titlespacing{\subparagraph}{0mm}{0mm}{0mm}

%define header and footer
\usepackage{fancyhdr}
\pagestyle{fancy}

\fancyhead[RO]{\AUTHOR\hspace{4pt}|\hspace{4pt}\INSTITUTE}
\fancyhead[LO]{\TITLE}
\usepackage[style=iso]{datetime2}
\fancyfoot[RO]{\today}
\renewcommand\headrulewidth{0pt}
\renewcommand\footrulewidth{0pt}
\headsep = -2pt
\footskip = 0pt

% no vertical distribution
%% explanation: we copy the macro columnbreak to stdcolumnbreak
%% we now redefine columnbreak to always fill up null space and then execute the standard columnbreak.
\let\stdcolumnbreak\columnbreak
\renewcommand\columnbreak{\vfill\null\stdcolumnbreak}


\newcommand{\TITLE}{C++ Code Overview}
\newcommand{\AUTHOR}{Mona Panchaud}
\newcommand{\INSTITUTE}{Ostschweizer Fachhochschule}


\begin{document}
\begin{multicols*}{3}


\section{Calculator}
\subsection{Calc}
\begin{minted}[breaklines]{c++}
// Calc.hpp
#ifndef CALC_HPP
#define CALC_HPP

// <iostream> nur bei main() verwenden
#include <iosfwd>

auto calc(int x, int y, char symbol) -> int;

auto calc(std::istream& in) -> int;
#endif

// Calc.cpp
#include "Calc.hpp" // Eigener Header zuerst!

#include <stdexcept>
#include <istream> // <iostream> nur bei main()

auto calc(int x, int y, char symbol) -> int {
    switch(symbol) {
        case '+':
        return x + y;
        case '-':
        return x - y;
        case '*':
        return x * y;
        case '/':
        if(y == 0) {
            throw std::invalid_argument{"Division By Zero is not allowed"};
        }
        return x / y;
        case '%':
        if(y == 0) {
            throw std::invalid_argument{"Modulo By Zero is not allowed"};
        }
        return x % y;
        default:
        throw std::invalid_argument{"Unknown symbol"};
    }
}

auto calc(std::istream& in) -> int { // Gut
    int x{-1};
    int y{-1};
    char symbol{' '};
    if (in >> x >> symbol >> y) {
        return calc(x, y, symbol);
    }
    throw std::invalid_argument{"Invalid input"};
}
\end{minted}

\subsection{Sevensegment}
\begin{minted}[breaklines]{c++}
// Sevensegment.hpp
#ifndef SEVENSEGMENT_HPP_
#define SEVENSEGMENT_HPP_

#include <iosfwd>

auto printLargeDigit(int digit, std::ostream& out) -> void;

auto printLargeNumber(int number, std::ostream& out) -> void;

auto printLargeError(std::ostream& out) -> void;
#endif

// Sevensegment.cpp
#include "Sevensegment.hpp"

#include <cstddef>
#include <cstdlib>
#include <ostream>
#include <array>
#include <algorithm>
#include <vector>

const size_t segmentSize = 5;

std::array<std::string, segmentSize> const minus {
    {
        "   ",
        "   ",
        " - ",
        "   ",
        "   "
    }
}; // other arrays omitted for brevity


auto printLargeDigit(int i, std::ostream& out) -> void {
    auto digit = digits.at(i);
    std::ranges::copy(digit, std::ostream_iterator<std::string>{out, "\n"});
    // ist das gleiche wie:
    // std::ranges::for_each(digit, [&out](auto el) {
    //     out << el << '\n';
    // });
}

auto printLargeNumber(int largeNumber, std::ostream& out) -> void {
    std::vector<std::array<std::string, segmentSize>> largeNumberDigits{};
    int potentiallyNegativeNumber = largeNumber;
    largeNumber = abs(largeNumber);
    while (largeNumber > 0) {
        int digit = largeNumber % 10;
        largeNumberDigits.insert(
            largeNumberDigits.begin(),
            digits.at(digit));
        largeNumber /= 10;
    }
    if (potentiallyNegativeNumber < 0) {
        largeNumberDigits.insert(
            largeNumberDigits.begin(),
            minus);
    } else if (potentiallyNegativeNumber == 0) {
        largeNumberDigits.push_back(digits.at(0));
    }

    for (size_t i = 0; i < segmentSize; i++) {
        // auslagern
        std::ranges::for_each(largeNumberDigits, [&out, i](auto digit) {
            out << digit.at(i);
        });
        out << '\n';
    }
}

auto printLargeError(std::ostream& out) -> void {
    // auslagern
    for (size_t i = 0; i < segmentSize; i++) {
        std::ranges::for_each(error, [&out, i](auto errorLetter) {
            out << errorLetter.at(i);
        });
        out << '\n';
    }
}
\end{minted}

\subsection{Pocketcalculator}
\begin{minted}[breaklines]{c++}
// Pocketcalculator.hpp
#ifndef POCKETCALCULATOR_HPP
#define POCKETCALCULATOR_HPP_

#include <iosfwd>

auto pocketcalculator(std::istream& input, std::ostream& output) -> void;
#endif

// Pocketcalculator.cpp
#include "Pocketcalculator.hpp" // Eigener Header zuerst
#include "Calc.hpp"
#include "Sevensegment.hpp"

#include <istream>
#include <sstream>
#include <stdexcept>
#include <string>

auto pocketcalculator(std::istream &input, std::ostream &out) -> void {
    std::string calculationLine;
    while(std::getline(input, calculationLine) && input.good()) {

        std::istringstream calculationStream{calculationLine};

        try {
            auto result = calc(calculationStream);
            if(std::abs(result) <= 99999999) {
                printLargeNumber(result, out);
            } else {
                printLargeError(out);
            }
        } catch (std::invalid_argument const & e) { // Gut, per const& gefangen
            printLargeError(out);
        }
    }
}
\end{minted}

\section{Word}

\begin{minted}[breaklines]{c++}
// Word.hpp
#include <algorithm>
#include <cctype>
#include <compare>
#include <cstddef>
#include <istream>
#include <iterator>
#include <ostream>
#include <string>
#include <vector>

#ifndef WORD_HPP_
#define WORD_HPP_

namespace text {
// ignored indentation for practical reasons
class Word {
    std::string characters{};

public:
    Word();
    explicit Word(std::string);

    auto print(std::ostream& out) const -> void;

    auto operator<=>(Word const& right) const -> std::weak_ordering;
    auto operator==(Word const& right) const -> bool;
};

inline auto operator<<(std::ostream& out, Word const& word) -> std::ostream& {
    word.print(out);
    return out;
}

inline auto operator>>(std::istream& in, Word& word) -> std::istream& {
    std::string wordInput{};
    auto wordStarted = false;

    while (in.good()) {
        auto isValid = std::isalpha(in.peek());
        if (isValid) {
            wordInput.push_back(in.get());
        } else if (wordInput.empty()) {
            // consume characters at beginning of stream
            in.ignore();
        } else {
            break;
        }
    }

    if (!wordInput.empty()) {
        word = Word{wordInput};
    } else {
        in.get(); // Stream to eof state.
    }

    return in;
}

}
#endif

// Word.cpp
#include <algorithm>
#include <cctype>
#include <iterator>
#include <ostream>
#include <stdexcept>
#include <string>
#include <vector>

#include "Word.hpp"

namespace text {
    Word::Word() : Word("EMPTY") {}

    Word::Word(std::string word) {
        if (word.empty()) {
            throw std::invalid_argument{"Cannot create empty word"};
        }

        auto result = std::ranges::all_of(word.begin(), word.end(), [](auto letter) {
            return std::isalpha(letter);
        });
        if (result) {
            this->characters = word;
        } else {
            throw std::invalid_argument{"Cannot create word with non-alpha characters"};
        }
    }

    auto Word::print(std::ostream& out) const -> void {
        out << characters;
    }

    auto Word::operator<=>(Word const& right) const -> std::weak_ordering {
        auto cmp_icase = [](unsigned char x, unsigned char y)
        {
            return std::toupper(x) <=> std::toupper(y);
        };

        auto res = std::lexicographical_compare_three_way(
        characters.begin(),
        characters.end(),
        right.characters.begin(),
        right.characters.end(),
        cmp_icase
        );
        return res;
    }

    auto Word::operator==(Word const& right) const -> bool {
        return (*this <=> right) == std::weak_ordering::equivalent;
    }
}
\end{minted}

\section{IndexableSet}
\begin{minted}[breaklines]{c++}
#ifndef INDEXABLE_SET_HPP
#define INDEXABLE_SET_HPP

#include <cstdlib>
#include <functional>  // für std::less
#include <iterator>
#include <set>
#include <stdexcept>

template <typename T, typename COMPARE = std::less<T>>
class IndexableSet : public std::set<T, COMPARE> {
public:
    using container = std::set<T, COMPARE>;

    // Take over other constructors
    using container::container;

    auto at(int pos) const -> T {
        if ((pos > 0 && pos >= this->size()) ||
        (pos < 0 && std::abs(pos) > this->size())) {
            throw std::out_of_range{"Position out of range"};
        }

        auto iterator = this->cbegin();
        if (pos < 0) {
            iterator = this->cend();
        }

        std::advance(iterator, pos);
        return *iterator;
    }

    auto operator[](int pos) const -> T { return at(pos); }

    auto front() const -> T {
        if (this->empty()) {
            throw std::out_of_range{"Set is empty"};
        }
        return at(0);
    }

    auto back() const -> T {
        if (this->empty()) {
            throw std::out_of_range{"Set is empty"};
        }
        return at(this->size() - 1);
    }
};

#endif
\end{minted}

\section{SafeVector}
\begin{minted}[breaklines]{c++}
template<typename T>
struct SafeVector : std::vector<T> {
    using container = std::vector<T>;
    using std::vector<T>::vector; // to inherit constructors!
    using size_type = typename container::size_type;
    using reference = typename container::reference;
    using const_reference = typename container::const_reference;
    reference operator[](size_type index) {
        return this->at(index);
    }
    const_reference operator[](size_type index) const {
        return this->at(index);
    }
    // should also provide front/back with empty() check
};
\end{minted}

\section{Garden Plan}
\begin{minted}[breaklines]{c++}
// Bed.hpp
#ifndef BED_HPP
#define BED_HPP

#include <iosfwd>
#include <string>

struct Bed
{
    // -> Classes with virtual members require a virtual Destructor
    virtual ~Bed() = default;

    auto print(std::ostream & output) const -> void;

    auto virtual pegs() const -> unsigned = 0;
    auto virtual ropes() const -> double = 0;
    auto virtual seeds() const -> double = 0;

    protected:
    auto virtual name() const -> std::string = 0;
};

#endif

// Rectangle.hpp
#ifndef RECTANGLE_HPP
#define RECTANGLE_HPP

#include "bed.hpp"

struct Rectangle : Bed
{
    // For structs everything is public by default!
    Rectangle(double shortSide, double longSide);

    auto pegs() const -> unsigned override;
    auto ropes() const -> double override;
    auto seeds() const -> double override;

private:
    auto name() const -> std::string override;

    double shortSide;
    double longSide;
};

#endif

// Rectangle.cpp
#include "rectangle.hpp"

Rectangle::Rectangle(double shortSide, double longSide)
: shortSide{shortSide}
, longSide{longSide}
{
}

auto Rectangle::pegs() const -> unsigned {
    return 4;
}

auto Rectangle::ropes() const -> double {
    return 2 * shortSide + 2 * longSide;
}

auto Rectangle::seeds() const -> double {
    return shortSide * longSide / 10.0;
}

auto Rectangle::name() const -> std::string {
    return "Rectangle";
}
\end{minted}


\end{multicols*}
\end{document}