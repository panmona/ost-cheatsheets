% Document
\documentclass[fontsize=8pt,paper=a4,paper=landscape,DIV=calc,]{scrartcl}
\usepackage[T1]{fontenc}
\usepackage{noto}
\usepackage[nswissgerman,english]{babel}
\renewcommand{\familydefault}{\sfdefault}

% Format
\usepackage[top=5mm,bottom=1mm,left=5mm,right=5mm,includehead]{geometry}
\setlength{\headheight}{\baselineskip}
\setlength{\headsep}{0mm}

\usepackage{multicol}
\setlength{\columnsep}{2mm}
\setlength{\columnseprule}{0.1pt}

% Color
\usepackage[svgnames]{xcolor}

% Math
\usepackage{amsmath}
\usepackage{amssymb}
\usepackage{amsfonts}
\newcommand*{\eq}{=}

%% Venn Diagrams
\usepackage{venndiagram}

%% Trees
%%% https://tex.stackexchange.com/a/425271
\usepackage{forest}
\forestset{
    ptree/.style={
        for tree={
            % grow'=0,
            % parent anchor=children,
            % child anchor=parent,
            grow'=east,
            parent anchor=east,
            child anchor=west,
            text width=7mm
        },
        before typesetting nodes={
            for tree={
                split option={content}{:}{content, my edge label},
            },
        },
    },
    my edge label/.style={
        if={
            > O_= {n'}{1}
        }{
            edge label={node [midway, below left, font=\tiny] {#1} }
        }{
            edge label={node [midway, above left, font=\tiny] {#1} }
        },
    }
}

% Code
\usepackage{listings}
\usepackage{minted}

\definecolor{ocherCode}{rgb}{1, 0.5, 0} % #FF7F00 -> rgb(239, 169, 0)
\definecolor{blueCode}{rgb}{0, 0, 0.93} % #0000EE -> rgb(0, 0, 238)
\definecolor{greenCode}{rgb}{0, 0.6, 0} % #009900 -> rgb(0, 153, 0)
\definecolor{teal}{rgb}{0.0, 0.5, 0.5}

\lstdefinestyle{code}{
    identifierstyle=\color{black},
    keywordstyle=\color{blue}\bfseries\small,
    ndkeywordstyle=\color{greenCode}\bfseries\small,
    stringstyle=\color{ocherCode}\ttfamily\small,
    commentstyle=\color{teal}\ttfamily\textit\small,
    basicstyle=\ttfamily\small,
    breakatwhitespace=false,
    breaklines=true,
    captionpos=b,
    keepspaces=true,
    showspaces=false,
    showstringspaces=false,
    showtabs=false,
    tabsize=2,
    belowskip=0pt,
    aboveskip=0pt
}

\lstset{
   extendedchars=true,
   basicstyle=\footnotesize\ttfamily,
   tabsize=2,
   breaklines=true,
   showspaces=false,
   showtabs=false
   showstringspaces=false,
   style=code
}

%% https://tex.stackexchange.com/a/536018
%% Allow for German characters in lstlistings.
\lstset{literate=
    {Ö}{{\"O}}1
    {Ä}{{\"A}}1
    {Ü}{{\"U}}1
    {ü}{{\"u}}1
    {ä}{{\"a}}1
    {ö}{{\"o}}1
}

%% Easier inline code
\newcommand{\cinline}[1]{\lstinline[language=c]|#1|}
\newcommand{\jinline}[1]{\lstinline[language=java]|#1|}
\newcommand{\csinline}[1]{\lstinline[language={[Sharp]C}]|#1|}

% Images
\usepackage{graphicx}
\graphicspath{{graphics/}}

% Links
\usepackage{hyperref}
\hypersetup{
    colorlinks=true,
    linkcolor=blue,
    filecolor=magenta,
    urlcolor=cyan,
}

% Smaller Lists
\usepackage{enumitem}
\setlist[itemize,enumerate]{leftmargin=3mm, labelindent=0mm, labelwidth=1mm, labelsep=1mm, nosep}
\setlist[description]{leftmargin=0mm, nosep}
\setlength{\parindent}{0cm}

% Smaller Titles
\usepackage[explicit]{titlesec}

%% Color Boxes
\newcommand{\sectioncolor}[1]{\colorbox{black!60}{\parbox{0.97\linewidth}{\color{white}#1}}}
\newcommand{\subsectioncolor}[1]{\colorbox{black!50}{\parbox{0.97\linewidth}{\color{white}#1}}}
\newcommand{\subsubsectioncolor}[1]{\colorbox{black!40}{\parbox{0.97\linewidth}{\color{white}#1}}}
\newcommand{\paragraphcolor}[1]{\colorbox{black!30}{\parbox{0.97\linewidth}{\color{white}#1}}}
\newcommand{\subparagraphcolor}[1]{\colorbox{black!20}{\parbox{0.97\linewidth}{\color{white}#1}}}

%% Title Format
\titleformat{\section}{\vspace{0.5mm}\bfseries}{}{0mm}{\sectioncolor{\thesection.~#1}}[{\vspace{0.5mm}}]
\titleformat{\subsection}{\vspace{0.5mm}\bfseries}{}{0mm}{\subsectioncolor{\thesubsection~#1}}[{\vspace{0.5mm}}]
\titleformat{\subsubsection}{\vspace{0.5mm}\bfseries}{}{0mm}{\subsubsectioncolor{\thesubsubsection~#1}}[{\vspace{0.5mm}}]
\titleformat{\paragraph}{\vspace{0.5mm}\bfseries}{}{0mm}{\paragraphcolor{\theparagraph~#1}}[{\vspace{0.5mm}}]
\titleformat{\subparagraph}{\vspace{0.5mm}\bfseries}{}{0mm}{\subparagraphcolor{\thesubparagraph~#1}}[{\vspace{0.5mm}}]

%% Title Spacing
\titlespacing{\section}{0mm}{0mm}{0mm}
\titlespacing{\subsection}{0mm}{0mm}{0mm}
\titlespacing{\subsubsection}{0mm}{0mm}{0mm}
\titlespacing{\paragraph}{0mm}{0mm}{0mm}
\titlespacing{\subparagraph}{0mm}{0mm}{0mm}

%define header and footer
\usepackage{fancyhdr}
\pagestyle{fancy}

\fancyhead[RO]{\AUTHOR\hspace{4pt}|\hspace{4pt}\INSTITUTE}
\fancyhead[LO]{\TITLE}
\usepackage[style=iso]{datetime2}
\fancyfoot[RO]{\today}
\renewcommand\headrulewidth{0pt}
\renewcommand\footrulewidth{0pt}
\headsep = -2pt
\footskip = 0pt

% no vertical distribution
%% explanation: we copy the macro columnbreak to stdcolumnbreak
%% we now redefine columnbreak to always fill up null space and then execute the standard columnbreak.
\let\stdcolumnbreak\columnbreak
\renewcommand\columnbreak{\vfill\null\stdcolumnbreak}
